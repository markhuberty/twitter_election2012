\documentclass{acm_proc_article-sp}
%\usepackage{caption}

\begin{document}
\title{Further experiments in multi-cycle forecasting of Congressional
  elections with social media}
\numberofauthors{1}
\author{\alignauthor
Mark Huberty\titlenote{Enormous credit and thanks are due to Len DeGroot of the
  Graduate School of Journalism at the University of California,
  Berkeley for hosting real-time publication of predictions during the
2012 election; and to Hillary Saunders for invaluable research
support. Additional thanks to F. Daniel Hidalgo, Jasjeet Sekhon, and
participants at the 2011 UC Berkeley Research Workshop in American
politics for helpful comments and feedback. The usual disclaimers apply.}\\
\affaddr{Travers Department of Political
  Science}\\
\affaddr{University of California, Berkeley}
\email{markhuberty@berkeley.edu}
}
% \thanks{This version prepared for the Midwest
%     Political Science Association Conference, April 2013, Chicago. First version: February 2011. Special
%     thanks to the Graduate School of Journalism at the University of
%     California, Berkeley for hosting the Voting with your Tweets
%     experiment for the 2012 Congressional elections. This project
%     would not have come off without the support, input, and hard work
%     of Len DeGroot and Hillary Sanders. Additional thanks to
%     to F. Daniel Hidalgo, Jasjeet
%     Sekhon, and participants at the 2011 Society for Political
%     Methodology meeting, and the Fall 2011 UC Berkeley Research Workshop in
%     American Politics, for helpful comments and feedback. All errors remain my
%     own.}

% \author{Mark Huberty\thanks{Travers Department of Political Science,
%     University of California, Berkeley. Contact:
%     \url{markhuberty@berkeley.edu}.}\\ PRELIMINARY DRAFT}
% \date{\today}
\maketitle
\begin{abstract}
  Interest in forecasting elections with social media has intensified
  in parallel with the increased availability and volume social media
  data. However, recent experiments have shown that ostensibly
  promising forecasting methods do not perform reliably. This paper
  provides further evidence of the instability of social media-based
  election forecasts. I first describe a new algorithmic experiment
  and discuss its performance in the 2010 and 2012 Congressional
  elections. I then illustrate the shortcomings of other
  methods. Placebo tests illustrate the relatively low information
  content of social media-based election predictors. I conclude with
  confirmation of observed problems with social media data under
  different sampling and electoral conditions. 
\end{abstract}
\section{Introduction}
\label{sec:introduction}

Twitter has become as increasingly prominent media source in political campaigns
\cite{economist2009,harden2009}. This popularity has led to a series
of attempts at predicting predict election outcomes from Twitter
message volumes, timing, and content. Most of those efforts report
successful predictions. However, these accuracy estimates rely on
back-casting the same election used to train and validate the
prediction algorithms. Consequently, their potential utility in future elections
remains unclear.

This paper reports an experiment in real-time, multi-cycle election forecasting for the
United States House of Representatives using Twitter data.
We show that relatively simple assumptions about the content of the
Twitter feed enable highly accurate election forecasts. For the 2012
election, forecasts for districts with either Democratic or Republican
Party incumbents correctly predicted 85-90\% of races several weeks
before the election. Forecasts for open seats in districts drawn after the 2010
 were accurate 67\% of the time. Forecast vote shares also correlated
well with actual vote shares.

We provide three caveats that inform against over-reading these
results. First, predictor behavior indicates that algorithms learned a
baseline by discovering the identity of the district incumbent, and
then adjusting predictions for that district from that baseline. That
strategy risks regression to the baseline over time, suggesting a
limited utility horizon. Second, we show that Twitter now reproduces
behavior from other media: incumbents receive radically more attention
on Twitter than challengers. Hence, as Twitter continues to mature, it
may come to reflect known biases in traditional media, rather than
acting as a novel and democratic avenue for influencing or measuring
political sentiment. Third, Twitter users and communication patterns
among them display significant partisan polarization and homophily, in
ways not necessarily correlated with empirical patterns of
partisanship. Hence Twitter's convenience sample poses problems for
accurately representing the electorate. These caveats point to future
problems to address in building viable forecasting algorithms from
social media streams.


\section{Prior forecasting results}
\label{sec:prior-forec-results}

Twitter has become a popular medium for both studying online political
behavior. As an information-push medium,
tweets promise an unvarnished, if also unstructured, look into
individuals' political attitudes. Pursuing this possibility, 
\cite{tumasjan2010election} show that mere counts of tweets provide an
accurate polling medium for party election outcomes in
Germany. In the United Kingdom, \cite{tweetminster2010} claim to have
predicted the 2009 General Election outcomes with 90\% accuracy at the
national level, and with 69\% accuracy at the seat
level. \cite{o2010tweets} show
that Twitter traffic about American presidential candidates correlated with
polling outcomes in 2008, and may provide a leading indicator of polling
performance. \cite{bermingham2011using} conclude that a mixed approach
using both sentiment analysis and message volume may offer marginal
advantages over volume-based methods alone. But their result,
estimating outcomes for the 2011 Republic of Ireland general election,
does worse than traditional polls as measured by the mean absolute
error between their predicted vote share and actual outcomes. \cite{sang2012predicting} use Twitter sentiment and population
measures to forecast the 2011 Dutch senate elections, though their
best results depend on ad-hoc re-weighting with polling information and
are thus unsatisfying. Most
recently, \cite{digrazia2013} show that a very simple measure of candidate media
attention on Twitter provides predictive value for their election vote
outcomes. 

The lack of credible data on the demographics and demographic trends
of politically-active Twitter users throws further doubt on these
results. \cite{mislove2011understanding} provide preliminary data
suggesting that American Twitter users are disproportionately male,
urban, and coastal compared with the American
population. \cite{conover2011} provide evidence from the 2010
Congressional election that these users organize themselves in
communities with significant partisan homophily: Twitter users may
talk about their anti-partisans, but they converse largely with their
co-partisans. Perhaps, of course, these demographic difficulties could
be overcome via suitable weighting strategies. But Twitter continues
to see very rapid growth and change in its user
base \cite{gwi2013,thenextweb2013}. Instability in either the demographic makeup
or pattern of Twitter use among politically-interested Twitter users
could quickly erode the utility of estimators trained on historic
data.

\cite{gayo2011limits} raise this issue and others to
cast doubt on reported successes at using Twitter
to forecast elections. They point out that cases of successful
prediction are almost entirely dominated by back-casting elections
that have already occurred, and thus have uncertain future-predictive
value; that the data gathering, cleaning, and prediction routines are
rarely released as reproducible results; and that predictions are
rarely compared to suitable baselines for accuracy. For the latter, he
specifically notes that predictions are rarely benchmarked to rates of
incumbent re-election instead of pure chance. Hence, while these
results are suggestive, they leave the long-term performance of social
media-based election predictors uncertain.


\section{Design and data gathering}
\label{sec:design-data-gath}

Addressing some of these critiques, we present an open, fully
out-of-sample test of Twitter-based election forecasting across
multiple election cycles. This test is designed to address the
problems of transparency, back-casting, and underlying user base
stability raised by critics of past election forecasting results. The
forecasts presented here forecast the 2012 United States House of
Representatives elections using machine learning algorithms trained on
the 2010 House elections. Using the 2012 election as a true
out-of-sample test of predictive performance addresses several of the
problems raised by \cite{gayo2011limits}. Forward-looking prediction
addresses both the stability of Twitter information dynamics over
time, and the lack of transparency in past attempts at election
prediction.\footnote{All forecasts reported here were published in
real time at \texttt{http://californianewsservice.org/category/tweet-vote/}.  All
algorithms and other code are available at \texttt{https://github.com/markhuberty/twitter\_election2012}.}

The differences between the 2010 and 2012 elections make this
particular experiment an very strong test. Three differences in
particular stand out: the shift from a midterm to a Presidential year
election; the post-2010 redistricting; and the pronounced 2010
Republican swing. Each of these three differences should challenge the
portability of an estimator across election cycles. Addressing the
transparency critique, all data acquisition, cleaning, and prediction
logic was released ahead of time, and all predictions were posted in
real time during the general election campaign.\footnote{We note one departure from this commitment to
  pre-release: examination of predictions post-hoc showed that an
  off-by-one error in one step of the cleaning process led to
  mis-alignment of data with districts about halfway through the general
  campaign. This led to a spurious collapse in predictor accuracy. The
  source of the error remains unclear. Fixing this error resolved the
  accuracy issue. The actual prediction algorithms used remained
  unchanged.}

\subsection{Data acquisition and cleaning}
\label{sec:data-acquisition}

Data acquisition used the Twitter search API to return messages
mentioning the full name of Democratic or Congressional candidates
running in contested races..\footnote{Data acquisition began on September 12
2010; and on September 1, 2012. Dates were timed to fall after as many
primary elections as possible.} Queries were run nightly for each day
in the general election campaign. Table
\ref{tab:volume-by-party-inc} provides summary statistics on total
message volumes per candidate. As is immediately apparent, the observed
Twitter volume per candidate exploded between 2010 and 2012. The query
procedure generated approximately 260,000 messages for 313 districts
in 2010, the same procedure collected 1.3 million messages for 369
districts in 2012.  %% Double check this count (esp districts)

Data cleaning retained only
data for districts wherein both candidates had Twitter
content. Further data cleaning removed extraneous messages resulting
from name homonyms, and excluded obviously irrelevant non-political
data. Extraneous content was identified by modeling tweet content
using a Latent Dirichlet Allocation topic model applied to
``documents'' created by aggregating all tweets relevant to each
district. Tweets were subsequently excluded based on terms discovered
via those models.\footnote{Football- and baseball-related messages were
  common sources of irrelevant data. Indicative terms included
  \texttt{mlb}, \texttt{yankees}, and \texttt{stephen smith} (a name
  homonym for both a sports journalist and a candidate for California
  district 34).} Data cleaning reduced the overall message volume
by approximately 25,000 in 2010; and by 200,000 in 2012.

The cleaned data demonstrate the first significant result presented in
this paper: Twitter volumes are strongly biased in favor of
incumbents. As figure \ref{fig:cand-msg-volume} shows, incumbents
received significantly greater attention on Twitter than challengers.
The detailed breakdown in table \ref{tab:volume-by-party-inc} shows
that a Democratic incumbent received approximately 33\% more messages
than their Republican challenger in 2010; and nearly three times more
in 2012. Similar results obtain for Republican incumbents. This
imbalance suggests why volume-based forecasting algorithms (e.g.,
\cite{digrazia2013,tumasjan2010election,bermingham2011using}) work:
candidates' message volumes echo an ingrained bias towards incumbents
that manifests itself across a variety of measures (fund raising,
conventional media attention, name familiarity, etc), and which
correlates well with high incumbent rates of success. 

% \begin{figure}[ht]
%   \centering
%   \includegraphics[width=\textwidth]{../figures/plot_daily_party_volume}
%   \caption{This figure shows the daily aggregate message volume by 
% party for each day of data gathering up to Election Day 2012.}
%   \label{fig:daily-msg-volume}
% \end{figure}

\begin{figure}[ht]
  \centering
  \includegraphics[width=0.9\columnwidth]{../figures/plot_raw_cand_volumes}
  \caption{Incumbents receive far more attention from Twitter users
    than challengers. This figure shows the comparative message volume for
    candidate pairs in each district. Colors indicate the party of the
    district incumbent. The diagonal line illustrates where points would fall if both candidates in a district received equal message volume.}
  \label{fig:cand-msg-volume}
\end{figure}

\begin{figure}[ht]
  \centering
  \includegraphics[width=0.9\columnwidth]{../figures/plot_msg_volume_vote_spread}
  \caption{Competitive districts receive more attention on
    Twitter. This figure shows how total district message volume
    varies with district competitiveness. Elections decided around the
  50\% cut point receive orders of magnitude more attention from
  Twitter users than less competitive races.}
  \label{fig:msg-volume-vote-spread}
\end{figure}

% latex table generated in R 2.15.3 by xtable 1.7-1 package
% Tue Jun 11 16:27:14 2013
\begin{table}[ht]
\centering
\begin{tabular}{lllrrr}
  \hline
Year & Party & Inc. Party & Median & Mean & Std. Dev. \\ 
  \hline
2010 & D & D & 123.0 & 566.6 & 2432.8 \\ 
  2010 & D & R & 18.0 & 88.8 & 256.4 \\ 
  2010 & R & D & 91.0 & 275.3 & 790.0 \\ 
  2010 & R & R & 51.0 & 345.1 & 1036.6 \\ 
  2012 & D & D & 585.0 & 2001.9 & 6679.9 \\ 
  2012 & D & O & 415.0 & 1338.7 & 2244.5 \\ 
  2012 & D & R & 144.0 & 818.4 & 2774.5 \\ 
  2012 & R & D & 164.0 & 753.7 & 2118.9 \\ 
  2012 & R & O & 481.5 & 1975.5 & 4724.0 \\ 
  2012 & R & R & 507.0 & 2256.5 & 7502.3 \\ 
   \hline
\end{tabular}
\caption{Message volumes by party, district incumbency, and election. This table provides summary statistics for candidate Twitter message volume for the 2010 and 2012 elections. Incumbent party refers to the party of the district incumbent, regardless of whether the incumbent stood for re-election. 'O' refers to open districts created after the 2010 redistricting.} 
\label{tab:volume-by-party-inc}
\end{table}


\subsection{Data transformation for prediction}
\label{sec:data-transf-pred}

For use in prediction, the data were transformed into a vector-space
representation of text in the following steps:
\begin{enumerate}
\item URLs, punctuation, English stopwords, and non-ASCII characters were removed and
  message case was standardized
\item Candidate names were replaced with party-specific placeholders
\item Officeholder names (the Speaker of the House, the Senate
  Majority Leader, and the President) were replaced with
  office-specific placeholders,
\item A message-level term-frequency matrix,
  was consolidated into a district-level matrix by summing term
  frequencies for each row belonging to a distinct district. In 2010,
  terms were restricted to those appearing in more than 1\%, and less
  than 99\% of districts. The same vocabulary was then used to build
  the 2012 prediction matrices. Message-level term frequencies were
  weighted by creation date prior to summation. Win-loss data were
   weighted linearly relative to election
  day. Vote share 
  predictors used the raw term-frequency data. These weights were
  selected based on their relative performance in out-of-sample tests
  on the 2010 data. The district-level term-frequency matrix was row-normalized.
\end{enumerate}

\subsection{Prediction algorithms}
\label{sec:pred-algor}

Prediction algorithms were developed based on 313 districts from the 2010 House of
Representatives election. An ensemble machine learning
algorithm \cite{van2007super} was trained on a random 90\% subset of
these districts, using actual election outcomes.\footnote{For discrete
  prediction, the ensemble
included variants on: lasso, support vector machines with various
kernel parameters, and random forests with various tuning
parameters. For vote share prediction, this ensemble was expanded to
include boosted regression, sparse partial least squares, step
regression, ridge regression, and multivariate adaptive splines. Algorithms were selected for their capacity to deal with
high-dimensional, very sparse data.} Accuracy was
estimated against the held-out districts. Estimates showed that binary
prediction did best when term frequencies received uniform time
weights; and that vote share prediction did best when term frequencies
received weights inverse linear in the number of days prior to
election date. These algorithms and the dictionaries of their
predictor terms were then used to generate forecasts for each day of
the 2012 general election campaign.

\section{Results}
\label{sec:results}

We discuss four results:
\begin{enumerate}
\item Election predictions were reasonably accurate given the relative
  simplicity of the predictors. This accuracy persisted even for
  estimates made far in advance of the election.
\item Examination of algorithm performance shows that predictions are
  made in reference to terms signaling the party of the incumbent; and
  then adjusted from that basis. This presents the danger of
  algorithms regressing to the incumbency baseline over time.
\item User and message content displays significant partisan biases
  that may render Twitter messages suspect as a means of measuring
  social sentiment
\item Partisan communities in Twitter display significant partisan
  homophily, consistent with results obtained under very different
  sampling and partisan ranking assumptions
\end{enumerate}

\subsection{Vote share predictions}
\label{sec:predictions}

Figure \ref{fig:prediction-corr-bydate} summarizes the predictive
accuracy for both vote share and win/loss predictions. 2010 results
back-casted election outcomes correctly at rates far exceeding the
rate of incumbent success in this sample. For 2012, some regression to
the incumbent win rate was observed: both the win-loss and vote share
algorithms forecast the correct winner in 88\% of cases, approximating
the 90\% rate of incumbent re-election for districts in our sample.

Much of the mis-prediction error can be attributed to systemic
predictor bias against Democratic candidates. As figure
\ref{fig:corr-voteshare} shows, the vote share predictor
under-predicted Democratic vote shares in a large number of districts
that the Democrats ultimately won. In contrast, vote shares in districts
won by Republicans were forecast much more accurately. Table
\ref{tab:accuracy-by-incumbency} makes this clear for the overall
point estimates: Republican districts were successfully forecast at
rates equalling the incumbent success rate, while forecasts for
Democratic districts ran well behind the incumbency baseline. This outcome
confirms a concern about Republican bias in the training algorithm: to
the extent that the algorithm learned an implicit bias from the 2010
elections, in which the Republicans achieved a historic swing in party
control of House seats, it would carry this bias forward. However, we
note that such problems should be muted as data for additional
elections is added to the training pool. 



\begin{figure}[ht]
  \centering
  \includegraphics[width=0.9\columnwidth]{../figures/plot_corr_final_prediction_actual}
  \caption{Predicted vote shares correlate well with actual
    outcomes. This figure shows the correlation between predicted and
    actual vote shares for the 2010 and 2012 elections. District
    labels are colored according to the party of the electoral winner.}
  \label{fig:corr-voteshare}
\end{figure}

\begin{figure*}[ht]
  \centering
  \includegraphics[width=1.75\columnwidth]{../figures/voteshare_winloss_correlation_bydate}
  \caption{Predictive algorithms provide accuracy leading
    forecasts. This figure shows the predictive accuracy of both the
    vote share and win/loss algorithms for each day in the general election campaign. Predictions are back-cast for the 2010 election, using the trained algorithm; and forecast for the 2012 election. Vote shares were converted to win/loss predictions at the 50\% cut point. Horizontal lines indicate the incumbent win rate for the districts in the total population of forecasted districts.}
  \label{fig:prediction-corr-bydate}
\end{figure*}

% latex table generated in R 2.15.3 by xtable 1.7-1 package
% Thu Jun  6 20:46:18 2013
\begin{table*}[ht]
\centering
\begin{tabular}{llrrrr}
  \hline
Year & Incumbent party & Voteshare accuracy & Win-loss accuracy & N & Incumbent win rate \\ 
  \hline
2010 & D & 0.82 & 0.86 & 200 & 0.69 \\ 
  2010 & R & 0.98 & 0.99 & 113 & 0.98 \\ 
  2012 & D & 0.86 & 0.85 & 159 & 0.90 \\ 
  2012 & O & 0.67 & 0.67 & 21 &  \\ 
  2012 & R & 0.90 & 0.90 & 189 & 0.91 \\ 
   \hline
\end{tabular}
\caption{Predictive accuracy by election and district
  incumbent. Voteshare accuracy computed as the share of winning candidates
  forecast to win $>$ 50\% of the two-party vote.} 
\label{tab:accuracy-by-incumbency}
\end{table*}



\begin{figure}[ht]
  \centering
  \includegraphics[width=0.9\columnwidth]{../figures/plot_accuracy_by_quantile}
  \caption{Predictive accuracy is biased towards Republican
    candidates. This figure illustrates how predictor performance
    breaks down according to the party of the district incumbent, and
    the competitiveness of the race. Notice that predictions for
    Democratic districts where Republicans won far outperformed
    Republican districts where Democrats won. Competitiveness is defined according to the ultimate Democratic vote share in the district. }
  \label{fig:accuracy-by-competitiveness}
\end{figure}

% \begin{figure}[ht]
%   \centering
%   \includegraphics[height=0.3\textheight]{../figures/plot_ae_by_quantile}
%   \caption{Vote share forecast errors are worse for close
%     districts. This figure shows the absolute vote share prediction
%     error broken down by
%     district incumbent and election competitiveness. Forecast error is
%     scaled by the margin of difference between the actual vote outcome
%     and 50\%. Competitiveness
%     is defined according to the ultimate Democratic vote share in
%     the district.}
%   \label{fig:mae-by-competitiveness}
% \end{figure}


\subsection{Algorithms rely on incumbent cues as a baseline}
\label{sec:algor-behav-perf}

Results from multi-cycle forecasting thus evidence regression to
the incumbency baseline. Examination of the structure of the
forecasting algorithms suggests why. The use of
bigram algorithms permits partisan and officeholding cues to exist in
the same features. As figure \ref{fig:rf-term-importance} shows, those
features receive substantial weight in the random forest components of
the ensemble predictor. Additional terms then adjust from this incumbency baseline. The
importance of these linguistic cues in the predictor is confirmed by
attempting to re-train a predictive algorithm on data pre-cleaned to
eliminate these cues; predictive accuracy, even for back-casting,
declines significantly.

This is not, per se, a bad strategy: the high rate of incumbent
re-election in the US system makes the incumbency baseline ideal. But
the instability in salient political issues makes consistent
adjustment from that baseline difficult. For instance, ``4 hcr'' was a
top predictive term for the vote share predictor. It played a
significant role in the 2010 elections, which followed closely on the
passage of healthcare reforms. Two years later, however, jobs and
fiscal policy played a much more salient role. Unstable issue salience
complicates attempts at topic-driven forecasting such as was attempted
here. 

\begin{figure}[ht]
  \centering
  \includegraphics[width=0.9\columnwidth]{../figures/plot_rf_algorithm_term_importance}
  \caption{Prediction algorithms key in on incumbency
    indicators. This figure shows the estimated term importance for the random forest algorithm component
    of the vote share and win-loss ensembles. Term importance is
    estimated as the normalized change in predictive error upon
    random permutation of each term. Each panel shows the top
    30 terms by importance for the highest-weighted random forest
    member of the SuperLearner library.}
  \label{fig:rf-term-importance}
\end{figure}

\subsection{Twitter users are highly partisan}
\label{sec:part-cont-twitt}

Mapping from Twitter conversations to election outcomes presumes not
just linguistic stability, but also some stability in Twitter's
convenience sample. To investigate the dynamics of the
politically-engaged Twitter community further, we would like to know
both (1) the pattern of partisan alignment of users in our sample and
(2) patterns of communication among partisans. We estimate the
partisan alignment of Twitter users via a two-step approach. First, we
identify a set of Twitter hashtags with a known partisan alignment. In
this case, we choose \texttt{\#tcot} for conservatives and
\texttt{\#p2} for liberals. Following \cite{conover2011}, we then
index all hashtag-containing messages by their hashtags. Using this
index, we compute the Jaccard similarity for co-occurrence of each
unique hashtag with the known partisan tag. We retain separate lists
of conservative and liberal tags with a Jaccard index of greater than
0.01. We make these lists exclusive by discarding any confusion terms
appearing in both lists.

Based on these hashtags, we construct a partisan score, $p_u$ for each
user that employed hashtags in their messages. The scores is computed
as the scaled difference between their use of conservative and liberal
tags, on the interval $[-1, 1]$ from most liberal to most
conservative. Formally:

 \begin{equation}
   \label{eq:pscore}
   p_u = \frac{\left|tags_{u,cons}\right| - \left|tags_{u, lib}\right|}{\left|tags_u\right|}
 \end{equation}

To score those users that do not use hashtags, we use the scored users
as an input to train a classifier based on the text of user
messages. Formally, we convert the continuous partisan score to binary
``liberal'' / ``conservative'' score around a cut point at zero. We
then train a naive Bayesian classifier to predict a user's
partisan class based on the language of their aggregated
messages.\footnote{Formally, we aggregate each user's tweets and
transform them into a term-frequency matrix with one row per
user. Term frequencies are TfIdf-weighted. Hashtags,
common English stopwords and terms used by fewer than 0.1\%, or more
than 99.9\%, of users were discarded. URLs were standardized to a
'URL' placeholder. All computation used the \texttt{sklearn} machine
learning suite for Python \cite{scikit-learn}.} Cross-validated
predictive accuracy averaged 92\% for the 2010 user population, and
86\% for the 2012 population. We then use the trained classifier to
predict the partisan alignment of all users, based on the text of
their messages.

These estimates raise two issues about the nature of the Twitter
user population and its utility as a sample for real-world political
forecasting. First, by this measure Republicans on Twitter are more
strongly partisan-aligned than Democrats. Figure
\ref{fig:user-pscore-distribution} shows that conservative-leaning
users score as more extreme partisans than liberal-leaning users.

Second, conservative-aligned users are better represented in the
population than liberals. Conservatives outnumbered liberals 6.8:1 in
2010, and 2.2:1 in 2012. This Republican dominance of
politically-active Twitter users contrasts with the general electorate,
in which the share of self-described Democrats has remained stable as
the percentage of self-described Republicans has fallen since 2000
\cite{pew2012}.



\begin{figure}[ht]
  \centering
  \includegraphics[width=0.9\columnwidth]{../figures/user_partisan_alignment_density}
  \caption{Twitter users are highly partisan. This figure shows the distribution of partisan alignment
    scores for users whose tweets contain hashtags. Scores were
    calculated on the $[-1 , 1]$ interval, where 1 is most
    conservative. Liberal scores are shows as absolute values to permit direct comparison.}
  \label{fig:user-pscore-distribution}
\end{figure}



% \begin{figure}[ht]
%   \centering
%   \includegraphics[width=\textwidth]{../figures/mt_graph_path_length_distribution}
%   \caption{Path length distribution between co- and anti-partisans within the \textbf{mention} graph. Path lengths are computed as the Dijkstra shortest path between a 1,000 random liberal-liberal, conservative-conservative, or liberal-conservative node pairs. }
%   \label{fig:rt-partisan-path-length}
% \end{figure}

\subsection{Twitter users display significant partisan homophily}
\label{sec:twitt-users-displ}


Third, Twitter communication networks display very high partisan
homophily, even if the content of the communication spans partisan
boundaries. Twitter users can signal partisan links of two
forms. First, one user may \textit{retweet} messages from
another. Second, one user may \textit{mention} another in a
message. Since retweets require a user to first have received the
message being retweeted, they tend to signal communication
links. Mentions, in contrast, may signal only attempts at
communication or citation. To put this another way, retweets may
signal conversation, while mentions signal references.

Quantitative measures of graph connectedness suggest a high degree of
partisan homophily: individuals talk to their co-partisans, but talk
about their anti-partisans. We measure this in two ways. First, we
exploit graph connectedness to determine how closely bound partisan
types are to each other and to their anti-partisans. Three connection
types are possible: two co-partisan (Liberal-Liberal and
Conservative-Conservative) and one anti-partisan
(Liberal-Conservative). For each type, we compute the Dijkstra
shortest path for 1000 randomly chosen pairs of each type. The resulting
mean path lengths, show in figure \ref{fig:path-lengths}, suggest a world in which
individuals are far more likely to be connected to, and to converse
with, their co-partisans, even if they may talk about both co- and
anti-partisans.

We confirm this impression by investigating the community structure of
the relationship graphs. Community detection used the Louvain
algorithm \cite{blondel2008fast}. For a community $C$ containing
users $U$ with partisanship scores $p \in \{-1, 1\}$, community partisanship was computed
as $p_C = \frac{1}{\left|U\right|}\sum_U p_u$. Figure
\ref{fig:partisan-subcommunities} 
illustrates the distribution of community partisan membership
for both the mention and retweet graphs. Figures
 illustrate the graph
structure linking these communities to each other.\footnote{Graph
  nodes correspond to single communities. Edges correspond to mentions
or retweets between community members. Node color corresponds to the
partisan mix of community members.} Again, we see that users tend
to speak to one another, while talking about a more uniformly
distributed set of partisans. The retweet graphs, signalling
communication, cluster by partisan membership; while the mention
graphs show dense connections between communities of different
partisan content. These results are consistent with
\cite{conover2011}, under completely different sampling procedures. 



% \begin{figure}[ht]
%   \centering
%   \includegraphics[angle=90, width=0.7\textheight]{../figures/user_rt_largest_cc_mst_2010.png}
%   \caption{Network representation of \textbf{retweet} connections between
%     users for \textbf{2010}. Colors represent a user's partisan alignment towards the
%     Republicans (reds) or Democrats(blue). Labels illustrate users with
%     very high degree centrality. This representation uses the maximum
%     spanning tree of the largest connected component of the retweet graph.}
%   \label{fig:rt-largest-cc-mst-2010}
% \end{figure}


% \begin{figure}[ht]
%   \centering
%   \includegraphics[angle=90, width=0.7\textheight]{../figures/user_rt_largest_cc_mst_2012.png}
%   \caption{Network representation of \textbf{retweet} connections between
%     users for \textbf{2012}. Colors represent a user's partisan alignment towards the
%     Republicans (reds) or Democrats(blue). Labels illustrate users with
%     very high degree centrality. This representation uses the maximum
%     spanning tree of the largest connected component of the retweet graph.}
%   \label{fig:rt-largest-cc-mst-2012}
% \end{figure}

\begin{figure}[ht]
  \centering
  \includegraphics[width=0.9\columnwidth]{../figures/plot_rt_mt_pathlengths_2010_2012}
  \caption{Conservatives are more tightly connected to each other via
    conversation than liberals. This figure shows the 95\% confidence interval of the mean
    Dijkstra path length between co- and anti-partisans in the
    retweet and mention graphs. Path lengths were computed for 1000 randomly chosen pairs of each type.}
  \label{fig:path-lengths}
\end{figure}

% \begin{figure}[ht]
%   \centering
%   \includegraphics[width=0.9\columnwidth]{../figures/mt_rt_graph_path_length_distribution}
%   \caption{Path length distribution between co- and anti-partisans. Path lengths are computed as the Dijkstra shortest path between a 1,000 random liberal-liberal, conservative-conservative, or liberal-conservative node pairs. }
%   \label{fig:partisan-path-length}
% \end{figure}

% \begin{figure}[ht]
%   \centering
%   \includegraphics[angle=90,
%   width=0.7\textheight]{../figures/user_mt_largest_cc_mst_2010.png}
%   \caption{Network representation of \textbf{mention} connections between
%     users for \textbf{2010}. Colors represent a user's partisan alignment towards the
%     Republicans (reds) or Democrats(blue). Labels illustrate users with
%     very high degree centrality. This representation uses the maximum
%     spanning tree of the largest connected component of the mention graph.}
%   \label{fig:mt-largest-cc-mst-2010}
% \end{figure}

% \begin{figure}[ht]
%   \centering
%   \includegraphics[angle=90,
%   width=0.7\textheight]{../figures/user_mt_largest_cc_mst_2012.png}
%   \caption{Network representation of \textbf{mention} connections between
%     users for \textbf{2012}. Colors represent a user's partisan alignment towards the
%     Republicans (reds) or Democrats(blue). Labels illustrate users with
%     very high degree centrality. This representation uses the maximum
%     spanning tree of the largest connected component of the mention graph.}
%   \label{fig:mt-largest-cc-mst-2012}
% \end{figure}


%% NOTE: notice that for the communities, there's a baseline random
%% prob of membership if the communities were just drawn from the
%% existing distribution. So bootstrap that: draw 1000 samples of k
%% pairs and figure out what the distribution of membership looks
%% like. 
\begin{figure}[ht]
  \centering
  \includegraphics[width=0.9\columnwidth]{../figures/rt_mt_community_partisanship}
  \caption{Retweet communities display greater partisan homophily than
    mention communities. Retweet communities tend to contain
    communities with well-defined partisanship (either very liberal,
    or very conservative). Mention communities display more uniformly
    mixed partisan members.}
  \label{fig:partisan-subcommunities}
\end{figure}

% \begin{figure}[ht]
%   \centering
%   \includegraphics[width=\textwidth]{../figures/voteshare_prediction_corr_bydate}
%   \caption{Predicted and actual vote shares by district for 2010 and 2012. For 2010, all numbers reflect back-casting using the trained algorithm. For 2012, predictions reflect the forecasted Democratic vote share on election day. }
%   \label{fig:predicted-actual-voteshare-bydistrict}
% \end{figure}

\begin{figure*}[ht]
  \centering
  \begin{minipage}[h]{0.45\linewidth}
    \begin{center}
      \includegraphics[width=\textwidth, clip=true, trim=0 2cm 0 2cm]{../figures/mt2010_community_graph}
      Mentions 2010
    \end{center}

  \end{minipage}
  \begin{minipage}[h]{0.45\linewidth}
    \begin{center}
      \includegraphics[width=\textwidth, clip=true, trim=0 2cm 0 2cm]{../figures/mt2012_community_graph}
      Mentions 2012
    \end{center}
  \end{minipage}\\
  \begin{minipage}[h]{0.45\linewidth}
    \begin{center}
      \includegraphics[width=\textwidth, clip=true, trim=0 2cm 0 1cm]{../figures/rt2010_community_graph}
      Retweets 2010
    \end{center}
  \end{minipage}
  \begin{minipage}[h]{0.45\linewidth}
    \begin{center}
      \includegraphics[width=\textwidth, clip=true, trim=0 2cm 0 1cm]{../figures/rt2012_community_graph}
      Retweets 2012
    \end{center}
  \end{minipage}
\caption{Retweet and mention patterns among partisan communities on
  Twitter. Nodes in each figure correspond to communities in the
  retweet or mention graph as discovered by the Louvain
  algorithm. Node color reflects the mean partisan makeup of the
  community. Edges are weighted by the degree of interconnectedness
  among users in each community.}
\label{}
\end{figure*}

\section{Discussion}
\label{sec:discussion}

The results presented here point to the difficulty of forecasting
beyond heuristics using 
convenience samples in social media. Prediction algorithms that showed
promise in the 2010 election saw significant regression by the 2012
cycle. Patterns of Twitter message volume displayed significant
ingrained biases towards incumbents. The partisan distribution of
Twitter users is significantly skewed, in ways that run contrary to
the evolving partisan structure of the American electorate. Patterns
of communication between users suggest that Twitter partisans are
fairly isolated: while they may speak of their partisan opponents,
evidence of strong connections between anti-partisans is weak. To the
extent that anti-partisans are connected in the communication graph,
it is the large media organizations, rather than individuals, who
bridge partisan communities. Each of these phenomena evolved between
the 2010 and 2012 elections, weakening the effect of whatever learned
compensation mechanisms had helped the 2010 predictions beat the
incumbency baseline.

These factors suggest future challenges for building valid forecasting
models. The forecasting strategy here implicitly learned the
incumbency baseline and then attempted to adjust off that baseline
based on patterns of linguistic content about candidates. Similar
strategies have proven useful in other contexts, such as forecasting
macroeconomic aggregates like unemployment rates
\cite{choi2012predicting}. But the results here suggest greater
instability in the underlying map between online and offline behavior
than exists in the macroeconomic context. This problem is not
new. Google's historically accurate search-based influenza forecasts
\cite{ginsberg2008detecting} broke down in both 2009 and 2012,
consequence of changes to the epidemiological profile of the flu virus
and its coverage in the media \cite{butler2013google}. Twitter has
far lower penetration in the general populace than Google search, and
greater volatility in its user base and usage patterns. Hence the
greater 



\section{Conclusion}
\label{sec:conclusion}

Social media-based election forecasts have proliferated in the last
several years. We provide what we believe to be the first multi-cycle
test of election forecasting with Twitter. Our results point to the
difficulty of building persistently accurate content-based forecasting
algorithms. Underlying changes in election dynamics, issue salience,
the Twitter user base, and patterns of Twitter use degrade the
performance of prediction algorithms across election cycles. Moreover,
the convenience sample of the Twitter user space is clearly unlike the
broader electorate in both partisan alignment and polarization. While
we might imagine learning a valid mapping from the Twitter user base
to the preferences of the broader electorate, the underlying
instability of that user base and its patterns of use will weaken the
mapping over time. While commercial uses of Twitter-based forecasting
face similar problems, they benefit from far more regular
opportunities to update than the two- or four-year American election
cycle. Likewise, while poll-of-polls based forecasts also must update
over time, they benefit from a far larger set of opportunities to
learn the relationship between polling aggregates and house effects on
the one hand, and election outcomes on the other, than
the history of Twitter or other social media permit.

Hence building effective and informative social media-based election
forecasting approaches will require a sustained commitment on par with
that required to build effective forecasting and turnout models from
polling. Obvious needs include better models of how Twitter behavior
relates to real-world phenoma like partisan identification and
turnout; better monitoring of how the underlying Twitter population
evolves between elections; and better measurement of the gap between
the politically-active Twitter population and the broader
electorate. 


% Here, something about how Twitter partisans are probably very likely
% to vote, etc, and so unlike the rest of the electorate. 


\bibliography{/home/markhuberty/bibs/twitter}
\bibliographystyle{plain}
\end{document}
