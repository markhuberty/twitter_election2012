\documentclass{sig-alternate-2013}
%\usepackage{caption}
\usepackage{multirow}
\usepackage{dcolumn}


\newfont{\mycrnotice}{ptmr8t at 7pt}
\newfont{\myconfname}{ptmri8t at 7pt}
\let\crnotice\mycrnotice%
\let\confname\myconfname%

\permission{Permission to make digital or hard copies of all or part of this work for personal or classroom use is granted without fee provided that copies are not made or distributed for profit or commercial advantage and that copies bear this notice and the full citation on the first page. Copyrights for components of this work owned by others than ACM must be honored. Abstracting with credit is permitted. To copy otherwise, or republish, to post on servers or to redistribute to lists, requires prior specific permission and/or a fee. Request permissions from permissions@acm.org.}
\conferenceinfo{CIKM'13,}{Oct. 27--Nov. 1, 2013, San Francisco, CA, USA.}
\copyrightetc{Copyright 2013 ACM \the\acmcopyr}
\crdata{978-1-4503-2263-8/13/10\ ...\$15.00.\\
http://--enter the whole DOI string from rightsreview form confirmation}

\clubpenalty=10000 
\widowpenalty = 10000


\begin{document}
\title{Multi-cycle forecasting of Congressional
  elections with social media}
\numberofauthors{1}
\author{\alignauthor
Mark Huberty\titlenote{Enormous credit and thanks are due to Len DeGroot of the
  Graduate School of Journalism at the University of California,
  Berkeley for hosting real-time publication of predictions during the
2012 election; and to Hillary Saunders for invaluable research
support. Additional thanks to F. Daniel Hidalgo, Jasjeet Sekhon, and
participants at the 2011 UC Berkeley Research Workshop in American
politics for helpful comments and feedback. The usual disclaimers apply.}\\
\affaddr{Travers Department of Political
  Science}\\
\affaddr{University of California, Berkeley}
\email{markhuberty@berkeley.edu}
}

\maketitle

\begin{abstract}
  Twitter has become a controversial medium for election forecasting. We provide further evidence that simplistic forecasting methods do not perform well on forward-looking forecasts. We introduce a new estimator that models the language of campaign-relevant Twitter messages. We show that this algorithm out-performs incumbency in out-of-sample tests for the 2010 election on which it was trained. That success, however, collapses when the same algorithm is used to forecast the 2012 election. We further demonstrate that volume-based and sentiment-based alternatives also fail to forecast future elections, despite promising performance in back-casting tests. We suggest that whatever information these simplistic forecasts capture above and beyond incumbency, that information is highly ephemeral and thus a weak performer for future election forecasts.
\end{abstract}

\section{Introduction}
\label{sec:introduction}

Social media promises a real-time, readily available data source with
which to introspect into the behavior of society at large. Many
studies have suggested that this data can augment
or supplant traditional measures of social attitudes like polling or
surveys. Elections in particular have seen 

This paper provides evidence of the difficulty of building effective
forecasts of US elections using social media. We present the results
of one of the first multi-cycle experiments in election prediction
using social media data. We show that algorithms trained on one
election perform poorly on a subsequent election, despite having
performed well on out-of-sample tests on the original election. We
provide evidence that this failure stems from the volatility of the
underlying data-generating process, even in an election system with
very short periods between elections such as in the U.S. House of
Representatives. We further show that this problem exists for other
otherwise promising forecasting methods as well. In short, simplistic
methods for forecasting elections from Twitter, even when their
results are correlated with election outcomes, provide relatively
little added benefit. 

\section{Social media as an election forecast}
\label{sec:social-media-as}

Twitter has become a popular medium for forecasting offline political
behavior from visible online behavior. As an information-push medium,
tweets promise an unvarnished, if also unstructured, look into
individuals' political attitudes. However, successful predictors have
proven ephemeral. Claims by \cite{tumasjan2010election} to have
successfully forecast the 2009 German elections using Twitter data,
were shown to be an artifact of researcher choices rather than
research design \cite{jungherr2012pirate}. Mixing sentiment analysis
and relative attention on Twitter to different candidates appeared
promising, but under-performed conventional polling in the 2011
Republic of Ireland general elections
\cite{bermingham2011using}. \cite{sang2012predicting} performed
somewhat better in the 2011 Dutch elections, but their best results
relied on ad-hoc re-weighting using the very polling information that
Twitter-based forecasts often aspire to replace. Finally,
\cite{o2010tweets} show that Twitter sentiment may correlate with
political polling, but nevertheless offers weak predictive power for
actual election outcomes.

These problems indicate a much broader problem for election prediction
via social media. Given the demographic
differences between the Twitter user base and the voting population
\cite{mislove2011understanding}, the inherent dynamism of political
language and activity, the partisan polarization of the Twitter
community \cite{conover2011}, and incentives for strategic behavior by
campaigns and motivated partisans, simple heuristics appear unlikely
to perform reliably as electoral predictors. At the very least, they
argue, valid claims for any prediction should require the analyst to
offer predictions ahead of time, clearly articulate how their
predictive algorithm works, and establish a reasonable
baseline--almost certainly not random chance--against which their
predictions should be judged \cite{metaxas2011not}. To date, very few
forecasts have done so.

U.S. federal elections may pose a particularly hard task for social
media-based forecasts. Most US elections are decided between only two
parties. Of those districts in which two candidates actually run, only
a fraction are actually competitive: incumbents win re-election more
than 85\% of the time, even in anti-incumbent years like 2010. Even
very close to the 50\% win / loss cutpoint, evidence suggests that
incumbents maintain advantages over their challengers
\cite{caughey2011elections}, something possibly untrue of other
political systems \cite{eggers2013validity}. Partisan control of
election district boundaries and other facets of election
administration may reinforce this outcome. Hence US elections pose a
very high bar: forecasts must beat a simple heuristic, incumbency,
that reliably forecasts future winners with high accuracy, even in
ostensibly competitive races.

\section{An N-gram forecasting model}
\label{sec:multi-cycle-forecast}

We first describe a new forecast based on the Twitter micro-blogging
service. This method differs from earlier methods by 
modeling the language of candidates' Twitter feeds, rather than using
simple counts or sentiment scores. Based on models built from the 2010
U.S. House of Representatives election, we generated forward-looking
forecasts for the 2012 election outcomes. In line with recommendations
from \cite{metaxas2011not}, we pre-released all data acquisition,
cleaning, and forecasting code. Forecasts themselves were published
daily ahead of the 2012 election.\footnote{All code for data acquisition,
  cleaning, and prediction was released at
  \texttt{https://github.com/markhuberty/twitter\_election2012}. All
  predictions were published in real-time at
  \texttt{http://californianewsservice.org/category/tweet-vote/}. The
  only exception to pre-release was a bugfix that altered the last
  several days of prediction prior to the 2012 election. An off-by-one
  error corrupted certain data and generated a spurious collapse in
  prediction accuracy. That collapse disappeared when we re-created
  the prediction inputs from raw data.} 


\subsection{Data acquisition}
\label{sec:data-acquisition}

All models used Twitter data acquired via the Twitter Search
API.\footnote{Note that this differs from other papers, which tend to
  use variants of the Twitter streaming API. That API may not
  replicate the actual content of Twitter well
  \cite{morstatter2013sample}. We use the search API in
  this instance because it provides a means of gathering all mentions
  of a candidate. Exceptions here include very
  high-volume candidates like Nancy Pelosi or
  Paul Ryan, whose daily mention volume exceeds 1500. In those cases, a candidate's data is
  right-censored. However, most of those cases concern races that
  weren't very competitive anyway.} Searches were performed each day,
looking for all mentions of every
known Republican or Democratic general election candidate in the prior
24 hours. We retrieved all tweets possible, up to the API limit of 1500 messages per query. Summary
statistics on Twitter message volume by candidate are shown in table
\ref{tab:volume-by-party-inc}. Data gathering began
on September 1 for the 2010 election, and on September 12 for the 2012
election. The final data sets included approximately 260,000 messages for
313 districts in the 2010 election, and 1.3 million messages for 369
districts in the 2012 election.

All data were filtered for noise prior to conversion to the bi-gram
bag-of-words model described above. Filtering attempted to identify
spam via a Latent Dirichlet Allocation topic model; tweets were
subsequently excluded based on the presence of terms found in
``noise'' topics. For example, sports-related messages were common
sources of irrelevant data. The sportscaster Stephen Smith shares a
name with a candidate for California's District 34. Consequently,
terms like \texttt{mlb} and \texttt{yankees} were signals for
politically-irrelevant data that were inadvertently captured because
of the name homonym. This cleaning reduced overall message volumes by
25,000 in 2010, and by 200,000 in 2012.

% latex table generated in R 2.15.3 by xtable 1.7-1 package
% Tue Jun 11 16:27:14 2013
\begin{table}[ht]
\centering
\begin{tabular}{lllrrr}
  \hline
Year & Party & Inc. Party & Median & Mean & Std. Dev. \\ 
  \hline
2010 & D & D & 123.0 & 566.6 & 2432.8 \\ 
  2010 & D & R & 18.0 & 88.8 & 256.4 \\ 
  2010 & R & D & 91.0 & 275.3 & 790.0 \\ 
  2010 & R & R & 51.0 & 345.1 & 1036.6 \\ 
  2012 & D & D & 585.0 & 2001.9 & 6679.9 \\ 
  2012 & D & O & 415.0 & 1338.7 & 2244.5 \\ 
  2012 & D & R & 144.0 & 818.4 & 2774.5 \\ 
  2012 & R & D & 164.0 & 753.7 & 2118.9 \\ 
  2012 & R & O & 481.5 & 1975.5 & 4724.0 \\ 
  2012 & R & R & 507.0 & 2256.5 & 7502.3 \\ 
   \hline
\end{tabular}
\caption{Message volumes by party, district incumbency, and election. This table provides summary statistics for candidate Twitter message volume for the 2010 and 2012 elections. Incumbent party refers to the party of the district incumbent, regardless of whether the incumbent stood for re-election. 'O' refers to open districts created after the 2010 redistricting.} 
\label{tab:volume-by-party-inc}
\end{table}

\subsection{Data characteristics}
\label{sec:data-characteristics}

The cleaned data illustrate two important results. First, Twitter volumes
are strongly biased in favor of incumbents. As figure
\ref{fig:cand-msg-volume} shows, incumbents received significantly
greater attention on Twitter than challengers.  The detailed breakdown
in table \ref{tab:volume-by-party-inc} shows that a Democratic
incumbent received approximately 33\% more messages than their
Republican challenger in 2010; and nearly three times more in
2012. Similar results obtain for Republican incumbents. This imbalance
suggests why volume-based forecasting algorithms (e.g.,
\cite{tumasjan2010election,bermingham2011using}) may work: candidates'
message volumes echo an ingrained bias towards incumbents that
manifests itself across a variety of measures (fund raising,
conventional media attention, name familiarity), and which
correlates well with high incumbent rates of success. Second, as shown
in figure \ref{fig:msg-volume-vote-spread} shows, highly competitive
elections--those decided by small margins around the 50\%
cutpoint--receive significantly more attention from Twitter users than
safe seats. Yet even there, incumbents continue to receive far more
attention than challengers. 

\begin{figure}[ht]
  \centering
  \includegraphics[width=0.9\columnwidth]{../figures/plot_raw_cand_volumes}
  \caption{Incumbents receive far more attention from Twitter users
    than challengers. This figure shows the comparative message volume for
    candidate pairs in each district. Fill indicate the party of the
    district incumbent. The diagonal line illustrates where points would fall if both candidates in a district received equal message volume.}
  \label{fig:cand-msg-volume}
\end{figure}


\begin{figure}[ht]
  \centering
  \includegraphics[width=0.9\columnwidth]{../figures/plot_msg_volume_vote_spread}
  \caption{Competitive districts receive more attention on
    Twitter. This figure shows how total district message volume
    varies with district competitiveness. Elections decided around the
  50\% cut point receive orders of magnitude more attention from
  Twitter users than less competitive races.}
  \label{fig:msg-volume-vote-spread}
\end{figure}


\subsection{Model description and assumptions}
\label{sec:model-descr-assumpt}

Our model departs from earlier attempts at Twitter-based election
prediction by modeling the language of the Twitter message feed
itself. Other estimates have used message volume or na\"ive sentiment
analysis. Either method assumes the meaning of Twitter language
\textit{a priori}: either through an ``all news is good news'' model
that ignores such language entirely, or by ignoring linguistic context
when applying sentiment lexicons. In contrast, we use one election as
an opportunity to learn predictive weights for language bigrams, on
the assumption that the salience of linguistic cues is relatively
static over short election cycles.

The model works as follows. In 2010, we gathered data on contested
U.S. House races between a Democrat (D) and a Republican (R). Messages
were case-standardized and stripped of English stopwords, URLs, and
non-ASCII characters. Candidates' proper names were standardized to
party-specific placeholders (``Rcanddummy'', ``Dcanddummy''). Each
message was then converted to a bi-gram bag-of-words term-frequency
representation. Term frequencies for all messages pertaining to a
single race--to one R-D candidate pair--were then summed to generate a
single bi-gram term-frequency vector for each election race. Terms
present in fewer than 1\%, or more than 99\%, of races, were
discarded. Vectors were normalized to sum to 1.

Using this bi-gram bag-of-words representation for districts, we
trained an ensemble machine learning algorithm on district-level
election outcomes for 2010. Outcomes were either continuous
(Democratic share of the two-party vote) or binary (Democratic win /
loss). Both algorithms relied on the SuperLearner ensemble supervised
learning algorithm \cite{van2007super,polley2010}, which trains a
library of standard machine learning algorithms against labeled
data.\footnote{The libraries in this case were specified to handle
high-dimensional, sparse data. For win-loss prediction, the library
included variants on: lasso, support vector machines with various
kernel parameters, and random forests with various tuning
parameters. For vote share prediction, this ensemble was expanded to
include boosted regression, sparse partial least squares, step
regression, ridge regression, and multivariate adaptive splines.}
Weights for each library member are learned via minimization of the
cross-validated risk of the ensemble forecast. Accuracy is bounded on
the low end by the best predictive performance of all the individual
algorithms in the library.

Examination of the structure of the learned model provides insight
into how the forecasting algorithm works. Examination of algorithm
weights in the final SuperLearner vote share algorithm showed that it
was dominated by variants of the random forest algorithm with
different tuning parameters. Figure \ref{fig:rf-term-importance}
illustrates the most influential terms within one of those random
forest variants. The final models are dominated by two sets of terms:
one set that signal for the party of the incumbent candidate, and
another that pick up on salient political issues. This is a sensible
general model: given the high rate of incumbent re-election in
U.S. politics, any reasonable model should start from the assumption
of incumbent victory, and adjust from that baseline given
politically-salient issues and other factors.

\begin{figure}[ht]
  \centering
  \includegraphics[width=0.9\columnwidth]{../figures/plot_rf_algorithm_term_importance}
  \caption{Prediction algorithms key in on incumbency
    indicators. This figure shows the estimated term importance for the random forest algorithm component
    of the vote share and win-loss ensembles. Term importance is
    estimated as the normalized change in predictive error upon
    random permutation of each term. Each panel shows the top
    30 terms by importance for the highest-weighted random forest
    member of the SuperLearner library.}
  \label{fig:rf-term-importance}
\end{figure}

\subsection{Model performance}
\label{sec:data-acqu-model}

All models were compared against the baseline rate of incumbent
re-election. Forecasting models were trained on 2010 data. Back-casting
2010 results on an out-of-sample data set suggested that the algorithms
could beat the incumbency baseline. As table
\ref{tab:accuracy-by-incumbency} shows, the algorithm beat the rate of
incumbent re-election in Democratic districts, and equalled it in
Republican districts.

% latex table generated in R 2.15.3 by xtable 1.7-1 package
% Thu Jun  6 20:46:18 2013
\begin{table*}[ht]
\centering
\begin{tabular}{llrrrr}
  \hline
Year & Incumbent party & Voteshare accuracy & Win-loss accuracy & N & Incumbent win rate \\ 
  \hline
2010 & D & 0.82 & 0.86 & 200 & 0.69 \\ 
  2010 & R & 0.98 & 0.99 & 113 & 0.98 \\ 
  2012 & D & 0.86 & 0.85 & 159 & 0.90 \\ 
  2012 & O & 0.67 & 0.67 & 21 &  \\ 
  2012 & R & 0.90 & 0.90 & 189 & 0.91 \\ 
   \hline
\end{tabular}
\caption{Predictive accuracy by election and district
  incumbent. Voteshare accuracy computed as the share of winning candidates
  forecast to win $>$ 50\% of the two-party vote.} 
\label{tab:accuracy-by-incumbency}
\end{table*}


\begin{figure}[ht]
  \centering
  \includegraphics[width=0.9\columnwidth]{../figures/voteshare_winloss_correlation_bydate}
  \caption{Predictive accuracy degrades between elections. This figure
    shows that algorithms capable of surpassing the incumbency
    baseline in 2010 were unable to do so in 2012. Predictions are back-cast for the 2010 election, using the trained algorithm; and forecast for the 2012 election. Vote shares were converted to win/loss predictions at the 50\% cut point. Horizontal lines indicate the incumbent win rate for the districts in the total population of forecast districts.}
  \label{fig:prediction-corr-bydate}
\end{figure}

However, this model fared significantly worse when generating
forward-looking forecasts of the the 2012 election. While it once
again equalled the rate of incumbent success in Republican districts,
forecasts for Democratic districts fell far short of the incumbency
re-election baseline. For open seats, without incumbents, created by
the post-2010 election redistricting, forecasts beat simple chance but
only predicted two-thirds of races correctly.

Thus the multi-cycle test presented here invalidates the assumption
implicit in the forecasting algorithm. The 2010 and 2012 elections
were fought over similar issues, such as healthcare regulation and
fiscal policy. But the underlying dynamics of Twitter use and
content--including the generation of newly salient issues and changing
representation of older issues--were not stable enough to permit an
algorithm that showed promise in one election to perform well in the
subsequent one. Instead, the incumbency portion of the forecast
remained valid, while the adjustment from that baseline fell apart.

\section{Comparative performance of alternative forecasts}
\label{sec:multi-cycle-forecast-1}

Other similarly promising Twitter-based forecasts may suffer from
similar problems \cite{gayo2011limits}. Given the data we have
available, we are also able to test the multi-cycle performance of
these methods. We provide two such tests: one based on relative
candidate volumes as used in \cite{tumasjan2010election} and
\cite{digrazia2013}; and the other based on na\"ive sentiment analysis
using the OpinionFinder sentiment corpus
\cite{wilson2005opinionfinder}. Both methods attempt to add to the
predictive power of incumbency in forecasting future elections. We
show that neither method does so when applied to out-of-sample
forecasts for future elections, as opposed to in-sample predictions
for the elections used for algorithm training.

\subsection{Volume-based forecasts}
\label{sec:volume-based-forec}

Volume-based forecasts assume a direct connection between relative
message volumes for candidates and their performance at the
polls. Following \cite{digrazia2013}, we construct a measure of
Twitter attention $T_R$ as the ratio of the Republican message volume $V_R$
to the total message volume $V_R + V_D$, as shown in equation
\ref{eq:more-tweets}. 

\begin{equation}
  \label{eq:more-tweets}
  R = \frac{T_R}{T_R + T_D}
\end{equation}

We then model election outcomes with OLS as specified in
equation \ref{eq:more-tweets-model}. Republican performance for an
election at time $t$ is modeled as a function of the Twitter proxy at
$t$ and the prior Republican vote share in that district $V_{t-1}$. This
explicitly models incumbency separately from the contemporaneous,
election-specific data derived from Twitter. 

\begin{equation}
  \label{eq:more-tweets-model}
  V_{R, t} = V_{R, t-1} + R
\end{equation}

Regression results are shown in table
\ref{tab:lm-volume-reg}. Consistent with \cite{digrazia2013}, we find
that $R$ remains significant even when explicitly accounting for
incumbency, though its magnitude declines substantially.\footnote{We
note that we do not use the same sampling method as they do, and so
the results here may not be directly comparable. However, the
substantive conclusion of the regression remains the same: a
candidate's share of Twitter mentions in a race remains significant
even when conditioning on a measure of incumbency. Furthermore, the
prior Congressional vote share is arguably a stronger measure of
incumbency than prior Republican Presidential candidate performance.}
Nevertheless, this estimator suffers the same regression to a weak
incumbency signal as the n-gram model discussed in section
\ref{sec:multi-cycle-forecast}. When back-casting the 2010 elections,
the volume-based forecast beat the rate of incumbent re-election. But
when used to forecast the 2012 election using contemporaneous data,
that forecast performed substantially worse than a simple incumbency
heuristic. Figure \ref{fig:lm_volume_model_graph} illustrates the
performance degradation. Incumbency out-performs all model variants in
2012. Furthermore, forecast under-performance was worst for those
elections that we care most about: those right around the 50\%
cutpoint. For races decided by a spread of 10 points or less (that is,
where one candidate's share of the two-party vote was in the interval
$(45, 55]$ percent), the complete model forecast only 63\% of the races
correctly in 2010, and only 53\% in 2012. Finally, the estimator
appears to gain little from the Twitter data itself. An OLS model
trained without $R$ performed nearly as well as the fully-specified
model in both 2010 and 2012. This is true whether measured by the RMSE
error for forecast vote share, or the binary win / loss accuracy rate.

\input{../tables/lm_volume_model}


%\input{../tables/volume_ratio_accuracy}

\subsection{Na\"ive sentiment analysis}
\label{sec:naive-sent-analys}

Finally, we implement a version of na\"ive sentiment analysis as a
forecasting proxy. Earlier studies \cite{o2010tweets,van2008good}
employed relatively simple sentiment analysis to generate either
polling proxies or predictive measures for campaign outcomes. Here we
use the OpinionFinder sentiment corpus \cite{wilson2005opinionfinder}
to assign sentiment scores to each candidate's tweets. Scores are
computed as the sum of positive (+1) and negative (-1) OpinionFinder
adjectives. A candidate's aggregate sentiment score is defined as $S =
\frac{pos}{pos + neg}$. For a two-candidate campaign, we define the
campaign sentiment ratio as $Sentiment = \frac{S_R}{S_R + S_D}$. Using
this metric, we fit a regression of the form:

\begin{equation}
  \label{eq:1}
  V_{R, t} = V_{R, t-1} + Sentiment
\end{equation}

Table \ref{tab:lm-sentiment-reg} summarizes the regression in its
fully-specified and component forms. We see that both prior vote share
and sentiment return significant predictors of the two-party vote
share. Once again, we see that the Twitter-based proxy remains
significant in the regression specification even when explicitly
modeling incumbency, though again its magnitude declines
substantially. However, those results do not translate into accurate
forward-looking predictions. Figure \ref{fig:lm_sentiment_model_graph}
summarizes both the win/loss accuracy and RMSE voteshare error for all
model specifications. We see that while back-casting the 2010 election
could beat the baseline rate of incumbent re-election, forecasting
2012 performed somewhat worse. Moreover, the sentiment proxy provided
no added predictive power: the model that used only vote share to
forecast 2012 performed as well as the complete model. Conversely, a
sentiment-only model performed substantially worse, and failed to beat
the incumbency baseline either when back-casting or
forecasting. Finally, performance was once again worst for the most
contested races: for races decided by a spread of 10 points or less,
the full model forecast only 65\% correctly.

\input{../tables/lm_sentiment_model}

\begin{figure}
  \centering
  \includegraphics[width=0.9\columnwidth]{../figures/volume_forecast_performance}
  \caption{Summary of volume forecast performance. This figure summarizes the sentiment forecast performance in 2010 and 2012. In all cases, the incumbency-based forecast performed at least as well as the Twitter-based forecasts. For predicting the winner alone, incumbency out-performed all other models. }
  \label{fig:lm_volume_model_graph}
\end{figure}
\begin{figure}
  \centering
  \includegraphics[width=0.9\columnwidth]{../figures/sentiment_forecast_performance}
  \caption{Summary of sentiment forecast performance. This figure summarizes the sentiment forecast performance in 2010 and 2012. In all cases, the incumbency-based forecast performed at least as well as the Twitter-based forecasts. For predicting the winner alone, incumbency out-performed all other models. }
  \label{fig:lm_sentiment_model_graph}
\end{figure}
%\input{../tables/tab_sentiment_accuracy}


\section{Discussion}
\label{sec:discussion}

These results add further weight to the argument that simplistic
measures of political sentiment or intent in Twitter traffic will not
suffice as valuable election forecasts. Each of the methods discussed
here generated promising results when back-casting elections. None of
them provided useful predictions for true out-of-sample
forward-looking forecasts. Forecasts were particularly inaccurate for
elections decided close to the 50\% win/loss cutpoint. These results
occurred despite a political system, the U.S. House of
Representatives, with short intervals between elections, and in which
adjacent elections are often fought over similar issues.

These results recommend against simplistic
election forecasts with Twitter. None of the methods used here made
vigorous attempts to account for the demographic or partisan
differences between the Twitter universe and the voting public. Nor
did they attempt to account for changes to that universe
itself. Instead, they all sought to find a useful mapping between a
snapshot of that universe, taken at one election, and actual election
outcomes. The results here suggest that both elections and the Twitter
universe more generally are sufficiently unstable as to quickly render
such maps invalid. 



\section{Conclusions}
\label{sec:conclusions}

We have provided three tests of heretofore promising approaches to
forecasting U.S. House of Representatives elections with
Twitter. Real-time forecasts based on n-gram patterns illustrated the
degradation of model performance between election. Use of the data
gathered for that experiment to test volume- or sentiment-based
forecasts showed that the same thing was true of those
methods. Examination of the data itself showed that Twitter data tends
to reproduce known biases towards incumbents in the U.S. political
system. Any predictive power above and beyond simply predicting that
the incumbent will win thus appears to come from over-fitting to
ephemeral phenomena unique to single elections. Real success at using
social media to forecast general political behaviors thus appears to
require much greater effort to detect and account for demographic,
political, and other differences between Twitter users and the broader
polity; and to do so continuously as both populations evolve and
change. Whether, after having done so, Twitter will fulfill its
promise as a simpler alternative to traditional polling, remains unclear.




% \begin{figure}[ht]
%   \centering
%   \includegraphics[width=0.9\columnwidth]{../figures/plot_corr_final_prediction_actual}
%   \caption{Predicted vote shares correlate well with actual
%     outcomes. This figure shows the correlation between predicted and
%     actual vote shares for the 2010 and 2012 elections. District
%     labels are colored according to the party of the electoral winner.}
%   \label{fig:corr-voteshare}
% \end{figure}


\bibliography{/home/markhuberty/bibs/twitter}
\bibliographystyle{plain}
\end{document}