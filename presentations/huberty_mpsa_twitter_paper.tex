\documentclass[11pt]{article}
\usepackage{natbib}
\usepackage[usenames, dvipsnames, svgnames, table]{xcolor}
\usepackage[dvipdfm,colorlinks=true,urlcolor=DarkBlue,linkcolor=DarkBlue,bookmarks=false,citecolor=DarkBlue]{hyperref}

\usepackage[pdftex]{graphicx}
\usepackage{fancyhdr}
\usepackage[T1]{fontenc}
\usepackage{palatino}
\usepackage[utf8]{inputenc}
%\usepackage[super]{nth}
\usepackage{setspace}
% \usepackage{placeins}
% \usepackage{subfigure}
% \usepackage{multirow}
\usepackage{rotating}
\usepackage{marvosym}  % Used for euro symbols with \EUR
\newcommand{\HRule}{\rule{\linewidth}{0.5mm}}
\usepackage{longtable} %% Allows the use of the longtable format produced by xl2latex.rb
\usepackage{lscape} %% Allows landscape orientation of tables
% \usepackage{appendix} %% Allows customization of the appendix properties
\setcounter{tocdepth}{1} %% Restricts the table of contents to the section header level entries only


\usepackage{geometry}
\geometry{letterpaper}
\usepackage{amsmath}
% \usepackage[stable]{footmisc}


\title{Voting with your Tweet:\\ Forecasting congressional elections
  with social media data\thanks{This version prepared for the Midwest
    Political Science Association Conference, April 2013, Chicago. First version: February 2011. Special
    thanks to the Graduate School of Journalism at the University of
    California, Berkeley for hosting the Voting with your Tweets
    experiment for the 2012 Congressional elections. This project
    would not have come off without the support, input, and hard work
    of Len DeGroot and Hillary Sanders. Additional thanks to
    to F. Daniel Hidalgo, Jasjeet
    Sekhon, and participants at the 2011 Society for Political
    Methodology meeting, and the Fall 2011 UC Berkeley Research Workshop in
    American Politics, for helpful comments and feedback. All errors remain my
    own.}}
\author{Mark Huberty\thanks{Travers Department of Political Science,
    University of California, Berkeley. Contact:
    \url{markhuberty@berkeley.edu}.}\\ PRELIMINARY DRAFT}
\date{\today}

\begin{document}
\maketitle
\doublespacing

\begin{abstract}
  This paper reports a large-scale out-of-sample experiment in
  election forecasting using Twitter data. Election outcomes for the
  2012 United States House of Representatives elections were forecast
  from Twitter data using algorithms trained on the 2010 House
  elections. Forecasts were 90\% accurate for districts with incumbent
  candidates; and approximately 75\% accurate for open
  seats. Predicted vote shares were highly correlated with actual vote
  shares. Further analysis shows that Twitter contains biases that may
  inform against further progress. Specifically, Twitter attention to
  candidates is highly skewed towards incumbents. Furthermore, Twitter
  communities display very high degrees of partisan homophily. We
  discuss the implications of both findings for future work.
\end{abstract}
\section{Introduction}
\label{sec:introduction}

The Twitter microblogging service has become increasingly popular for
political communication. This popularity has led to a series of
efforts attempting to predict election outcomes from Twitter message
volumes, timing, and sentiment. Most of those efforts, however, focus
on back-casting elections. Consequently, their potential utility in
future elections remains unclear. 

We report an experiment in real-time forecasting of elections for the
United States House of Representatives.\footnote{All forecasts
  reported here were published in real time at
  \url{http://californianewsservice.org/category/tweet-vote/}. All
  algorithms and other code are available at
  \url{https://github.com/markhuberty/twitter_election2012}.} We show
that relatively simple assumptions about the content of the Twitter
feed enable highly accurate election forecasts. Forecasts for
districts with either Democratic or Republican Party incumbents
averaged approximately 90\% accuracy several weeks before the
election. This was true in both close races and safe seats. Forecasts
for open seats--including new districts drawn after the 2010
census--averaged 75\% accuracy. Forecast vote shares also correlated
well with actual vote shares. 

We caution, however, against over-reading these results. Algorithm
behavior shows evidence of using Twitter language to discover the
identity of the incumbent in the race. Furthermore we show that
Twitter now reproduces behavior from other media: incumbents receive
radically more attention on Twitter than challengers. Hence, as
Twitter continues to mature, it may come to reflect known biases in
traditional media, rather than acting as a novel and democratic avenue
for influencing or measuring political sentiment.

\section{Prior work}
\label{sec:prior-work}



\section{Design and data gathering}
\label{sec:design-data-gath}

To resolve the out-of-sample forecasting problem, we implemented true
open-source, out-of-sample forecasting. Those forecasts were based on
the following process:

\begin{enumerate}
\item Gather Twitter data during the 2010 Congressional election
\item Build supervised learning algorithms to map 2010 election
  outcomes by district to the content of Twitter messages about
  district candidates
\item Gather new data during the 2012 Congressional election
\item Use the trained algorithms to forecast both the election winner
  and the Democratic vote share in real time
\end{enumerate}

By using the 2012 election as a true out-of-sample test of algorithm
performance, we gain traction on several problems. First, we can
estimate the stability of language over time. Second, we can estimate
how algorithms fare in very different elections. In this case, two
major changes occur between 2010 and 2012: from mid-term to
presidential-year elections; and from one set of Congressional
districts to another, consequence of the post-2010 Census
redistricting. 

\subsection{Data acquisition}
\label{sec:data-acquisition}

Data were acquired via the Twitter Search API. For each day during the
General Election campaign, we queried the Twitter Search API for the
full name of each candidate running for Congressional office. \footnote{Each general election campaign
  starts at a different date, consequence of variation in state
  primary schedules. In practice, we gathered data daily starting on
  September 1 2012.} Figure \ref{fig:daily-msg-volume} provides the
time-series message volume. Daily volumes averaged approximately ()
messages. Overall, we collected approximately 1.02 million individual
messages.

\subsection{Data cleaning}
\label{sec:data-cleaning}

Prior to analysis, we cleaned the raw data in four ways:
\begin{enumerate}
\item Only contested districts were included
\item Only districts where both candidates were represented were included
\item Name homonyms were omitted where possible
\item Obviously irrelevant data were excluded
\end{enumerate}

Name homonyms were closely correlated with irrelevant data. In
particular, sports-themed data related to both the end of the American
baseball season and the start of the American football season were
common. Both groups were found by modeling tweet content using a basic
Latent Dirichlet Allocation topic model, and excluding tweets and
users based on the terms discovered through those models.

The cleaned data show two important biases. First, Republican
candidates averaged more Twitter volume than Democrats. At the median,
the Republican candidate in a district received 28\% more message
volume than their Democratic challenger. Second, incumbents received
substantially more attention than challengers. At the median, an
incumbent was mentioned in three times as many messages (595 to 196)
as their challenger.



%% Table here

\begin{figure}[ht]
  \centering
  \includegraphics{../figures/plot_daily_party_volume}
  \caption{This figure shows the daily aggregate message volume by party for each day of data gathering up to Election Day 2012. }
  \label{fig:daily-msg-volume}
\end{figure}

\begin{figure}[ht]
  \centering
  \includegraphics{../figures/plot_raw_cand_volumes}
  \caption{This figure shows the comparative message volume for
    candidate pairs in each district. Colors indicate the party of the
    district incumbent. The diagonal line illustrates where points would fall if both candidates in a district received equal message volume.}
  \label{fig:cand-msg-volume}
\end{figure}

\subsection{Data transformation for prediction}
\label{sec:data-transf-pred}

For use in prediction, the data were transformed into a vector-space
representation of text in the following steps:
\begin{enumerate}
\item URLs, punctuation, and non-ASCII characters were removed
\item Candidate names were replaced with party-specific placeholders
\item All text was converted to lowercase
\item English stopwords were removed
\item Each individual tweet was converted into a term-frequency
  vector-space representation. For prediction, terms were restricted
  to a dictionary of bigrams corresponding to the variables used in
  the trained prediction algorithms
\item The resulting term-frequency matrix, with one row per message,
  was consolidated into a district-level matrix by summing term
  frequencies for each row belonging to a distinct district. For
  voteshare predictions, the message-level term-frequency values were
  first weighted linearly by their creation date relative to election
  day (e.g., a tweet that originated one week prior to election day
  would be weighted by a factor $\frac{1}{7}$). Binary (win-loss)
  predictors used the raw term-frequency data. These weights were
  selected based on their relative performance in out-of-sample tests
  on the 2010 data. 
\end{enumerate}

For real-time prediction, this process was repeated daily. 


\section{Results}
\label{sec:results}


\subsection{Predictions}
\label{sec:predictions}


\subsubsection{Win-loss predictions}
\label{sec:win-loss-predictions}

\subsubsection{Voteshare predictions}
\label{sec:votesh-pred}




\subsection{Social network behavior}
\label{sec:soci-netw-behav}

The data collected for the purposes of prediction also permit analysis
of the social behavior of politically-oriented Twitter users. 

\subsection{Constructing a retweet graph}
\label{sec:constr-retw-graph}

Retweets provide a very useful way of identifying connections between
users wherein we know that (1) one user follows another and (2) read
and forwarded a message. These data are thus potentially more useful
for considering social relationships than the Twitter follower graph,
which may be dominated by weaker ties. Nodes in the retweet graph
represent users; directed edges represent one user retweeting one or
more messages from another; edge weights represent the total retweet
count. To filter out users who may appear extraneously, we consider
only those users who generated (via their own tweets or retweets) a
message volume greater than the mean of all user volumes over the
entire election cycle. The resulting graph has 33,005 users and
136,888 edges. 



\subsection{Estimating user partisanship}
\label{sec:estim-user-part}

We estimate the partisan alignment of Twitter users via a two-step
approach. First, we identify a set of hashtags with a known partisan
alignment.\footnote{Hashtags here are individual terms of the form
  \texttt{\#word}, commonly used to tag tweets for easy retrieval and
  subject identification.} Beginning with a hashtag with a known
partisan alignment (\texttt{\#tcot} for Republicans and \texttt{\#p2}
for liberals), we find all messages where that seed tag occurs. We
then construct a list of all unique tags, and compute the Jaccard
index for the co-occurrance of each unique tag and the known partisan
tag. We retain tags with a Jaccard index of greater than 0.01 as tags
with a similar alignment as the seed tag. After identifying both
conservative and liberal tags, we eliminate any confusion terms that
appear in both lists. 

 Based on these hashtags, we construct a partisan score for each user,
 $p_u$, as the scaled difference between their use of conservative and
 liberal tags in their messages. Scores range from -1 (most liberal) to 1 (most conservative). The resulting distribution of partisan scores for users is shown in
figure \ref{fig:user-pscore}. Formally:

 \begin{equation}
   \label{eq:pscore}
   p_u = \frac{\left|tags_{u,cons}\right| - \left|tags_{u, lib}\right|}{\left|tags\right|}
 \end{equation}

However, not all users use hashtags. To
score those users that do not, we use the scored users as an input to
train a classification algorithm based on the text of user
messages. Formally, we convert the continuous partisan score to a
4-class rating from ``Very liberal'' to ``Very conservative''. We then
train a multinomial naive Bayesian estimator to predict a user's
partisan class based on the language of their aggregated
messages.\footnote{Formally, we aggregate each user's tweets and
  transform them into a term-frequency matrix. Term frequencies are
  weighted using a TfIdf rank. Terms used by fewer than 1\%, or
  more than 99\%, of users are discarded. All computation was done
  using the \textit{scikits-learn} machine learning package for Python.} The classifier was trained
on a random 90\% subset of this data. Estimation of its accuracy on
the held-out 10\% places the accuracy at greater than 70\%. We then
use the trained classifier to predict the partisan alignment of all
users in the retweet graph.

\begin{figure}[ht]
  \centering
  \includegraphics[angle=90,height=\textwidth]{../figures/user_partisan_alignment_byuser.pdf}
  \caption{This figure shows the partisan score for users whose
    messages contain hashtags, as computed using equation
    \ref{eq:pscore}. Scores range from -1 (most liberal) to 1 (most conservative)}
  \label{fig:user-pscore}
\end{figure}

Two issues immediately become apparent from these estimates. First, by
this measure Republicans on Twitter are more strongly partisan-aligned
than Democrats. Figure \ref{fig:user-pscore-distribution} shows that
conservative users are much more likely to score at the more extreme
end of the spectrum than liberal users. Second, among all users, acute
partisans are far better represented than moderates. Users with the
most extreme scores (-1 for liberals, or 1 for conservatives) account
for 72\% of all users whose partisan alignment could be directly
estimated from hashtags. 

\begin{figure}[ht]
  \centering
  \includegraphics[width=\textwidth]{../figures/user_partisan_alignment_density}
  \caption{This figure shows the distribution of partisan alignment
    scores for users whose tweets contain hashtags. Liberal scores are shows as absolute values to permit direct comparison with conservative scores. }
  \label{fig:user-pscore-distribution}
\end{figure}

\subsection{Descriptive results}
\label{sec:descriptive-results}

Figure \ref{fig:largest-cc-mst} illustrates the resulting retweet
graph, colored by the partisan alignment of the users. The figure
provides visual evidence of partisan polarization among the Twitter
user community: Republican-aligned users (shown in red) are weakly
connected to Democrat-aligned users (shown in blue). That visual
impression is confirmed by the graph's descriptive statistics. First,
co-partisans account for 89\% of all observed retweets. Second, within
the retweet graph, we are much more likely to find paths from any one
node to some other node if the two nodes are both liberal (22\% of
liberal node pairs share a path), or both conservative (17\%) versus
paths between anti-partisans (13\%). Third, as figure the shortest path between
any two nodes is significantly shorter for liberals (a mean 6 degrees
of separation) than for conservatives or anti-partisan pairs (a mean 7
degrees of separation). 

\begin{figure}[ht]
  \centering
  \includegraphics[angle=90, width=0.7\textheight]{../figures/user_rt_largest_cc_mst.png}
  \caption{Network representation of retweet connections between
    users. Colors represent a user's partisan alignment towards the
    Republicans (reds) or Democrats(blue). Labels illustrate users with
    very high degree centrality. This representation uses the maximum
    spanning tree of the largest connected component of the retweet graph.}
  \label{fig:largest-cc-mst}
\end{figure}

\begin{figure}[ht]
  \centering
  \includegraphics[width=\textwidth]{../figures/rt_graph_path_length_distribution}
  \caption{Path length distribution between co- and anti-partisans within the retweet graph. Path lengths are computed as the Djikstra shortest path between a 1,000 random liberal-liberal, conservative-conservative, or liberal-conservative node pairs. }
  \label{fig:rt-partisan-path-length}
\end{figure}

Hence the retweet graph paints a picture of heavily partisan political activity
on Twitter. Users are more likely to be extreme partisans; more likely
to connect with others of their own partisan makeup; and relatively
unlikely to converse across the partisan divide. 


\section{Discussion}
\label{sec:discussion}

The data presented here suggest that, for the 2012 Congressional election:

\begin{enumerate}
\item 
\item Incumbency and partisan alignment bias Twitter message volumes
  in consistent ways
\item User behavior appears to be highly partisan
\item Partisan homophily tends to dominate communication within
  Twitter social networks
\item  Partisans are more likely to establish ``strong'' ties with,
  and are more closely connected to,
  co-partisans than anti-partisans
\end{enumerate}



\section{Conclusions}
\label{sec:conclusions}



\end{document}
